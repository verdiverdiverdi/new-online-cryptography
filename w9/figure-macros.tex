% FIGURE COMMANDS
\definecolor{mBlue}{HTML}{22a1dc}
\definecolor{sBlue}{HTML}{009FE3}
\definecolor{sYellow}{HTML}{FFED00}
\definecolor{sRed}{HTML}{E30613}
\definecolor{sGreen}{HTML}{009640}
\definecolor{sPurple}{HTML}{440099}
\definecolor{mGreen}{HTML}{86cd30}
\definecolor{darkgreen}{HTML}{006400}
\definecolor{mLightBrown}{HTML}{e68a00}
\definecolor{lightred}{HTML}{FF9999}

%% COLOUR COMMANDS
\newcommand{\sBlue}[1]{\textcolor{sBlue}{#1}}
\newcommand{\sB}[1]{\sBlue{#1}}
\newcommand{\sRed}[1]{\textcolor{sRed}{#1}}
\newcommand{\sR}[1]{\sRed{#1}}
\newcommand{\sGreen}[1]{\textcolor{sGreen}{#1}}
\newcommand{\sG}[1]{\sGreen{#1}}
\newcommand{\sPurple}[1]{\textcolor{sPurple}{#1}}
\newcommand{\sP}[1]{\sPurple{#1}}
\newcommand{\sYellow}[1]{\textcolor{sYellow}{#1}}
\newcommand{\sY}[1]{\sYellow{#1}}

%% MATH MODE COLOR BOXES
\newcommand{\highlight}[2][yellow]{\mathchoice%
	{\colorbox{#1}{$\displaystyle#2$}}%
	{\colorbox{#1}{$\textstyle#2$}}%
	{\colorbox{#1}{$\scriptstyle#2$}}%
	{\colorbox{#1}{$\scriptscriptstyle#2$}}}%

% ClientAction and ServerAction
% Print a command executed by the client/server
% 1st argument: the text
\newcommand{\ClientAction}[1]{
	\node[right] at (\InitX, \Y) {#1};
}
\newcommand{\ServerAction}[1]{
	\node[left] at (\RespX, \Y) {#1};
}
\newcommand{\SharedAction}[1]{
	\node at ($1/2*(\InitX, \Y)+1/2*(\RespX, \Y)$) {#1};
}
\newcommand{\AdversaryAction}[1]{
	\node at ($1/2*(\InitX, \Y)+1/2*(\RespX, \Y)$) {\textcolor{red}{#1}};
}
% ClientToServer and ServerToClient
% Draws a message flow from client-to-server or server-to-client, with text above and below
% 1st argument (optional): line type, default ->
% 2nd argument: text above
% 3rd argument: text below
% Example: \ClientToServer{$Y$}{}
% Example: \ClientToServer[<->,double]{$Y$}{over an encrypted channel}
\newcommand{\ClientToServer}[3][->]{
	\NextLine[0.5]
	\draw[#1] (\ArrowLeft,\Y) -- node[above] {#2} node[below] {#3} (\ArrowRight,\Y) ;
	\NextLine[0.5]
}
\newcommand{\ServerToClient}[3][->]{
	\NextLine[0.5]
	\draw[#1] (\ArrowRight,\Y) -- node[above] {#2} node[below] {#3} (\ArrowLeft,\Y) ;
	\NextLine[0.5]
}
\newcommand{\ClientToAdversary}[3][->]{
	\NextLine[0.5]
	\draw[#1] (\ArrowLeft,\Y) -- node[above] {#2} node[below] {#3} (\ArrowCenter,\Y) ;
	\NextLine[0.5]
}
\newcommand{\ServerToAdversary}[3][->]{
	\NextLine[0.5]
	\draw[#1] (\ArrowRight,\Y) -- node[above] {#2} node[below] {#3} (\ArrowCenter,\Y) ;
	\NextLine[0.5]
}
\newcommand{\AdversaryQToClient}[3][->]{
	\NextLine[0.5]
	\draw[#1] (\ArrowCenter,\Y) -- node[above] {\textcolor{red}{#2}} node[below] {#3} (\ArrowLeft,\Y) ;
	\NextLine[0.5]
}
\newcommand{\AdversaryQToServer}[3][->]{
	\NextLine[0.5]
	\draw[#1] (\ArrowCenter,\Y) -- node[above] {\textcolor{red}{#2}} node[below] {#3} (\ArrowRight,\Y) ;
	\NextLine[0.5]
}
\newcommand{\AdversaryToClient}[3][->]{
	\NextLine[0.5]
	\draw[#1] (\ArrowCenter,\Y) -- node[above] {#2} node[below] {#3} (\ArrowLeft,\Y) ;
	\NextLine[0.5]
}
\newcommand{\AdversaryToServer}[3][->]{
	\NextLine[0.5]
	\draw[#1] (\ArrowCenter,\Y) -- node[above] {#2} node[below] {#3} (\ArrowRight,\Y) ;
	\NextLine[0.5]
}
\newcommand{\Encryption}[3][<->]{
	\NextLine[0.5]
	\draw[#1] (\ArrowRight,\Y) -- node[above] {#2} node[below] {#3} (\ArrowLeft,\Y) ;
	\NextLine[0.5]
}
% NextLine
% 1st argument (optional): amount of spacing to increment by, default 1.0
% Example: \NextLine
% Example: \NextLine[1.5]
\newcommand{\NextLine}[1][1.0]{
	\pgfmathparse{\Y+#1}
	\edef\Y{\pgfmathresult}
}
%
% stage separator line
%
\newcommand{\StageSeparator}[2][1.0]{
	\draw[very thick,dotted,blue] (\InitX,\Y+0.5) node[above=-0.1cm,anchor=north west] {\bf #2} -- (\RespX,\Y+0.5);
}
\newcommand{\Separator}[2][1.0]{
	\draw[very thick,dotted,blue] (\InitX,\Y+0.5) node[above=-0.1cm,anchor=north west] {\bf #2} -- (\RespX,\Y+0.5);
}
\newcommand{\StageRight}[1]{
	\node[above=-0.1cm,anchor=north east,StageSeparatorColor] at
	(\RespX,\Y+0.5) {\bf #1};
}

% TIKZ PICTURES

\tikzset{cloud/.pic={
		\node[cloud, cloud puffs=10.8,cloud puff arc=110, aspect=4, draw, text width=2cm
		] () at (0,0) {\tikzpictext};
	}}

\tikzset{database/.style={
		cylinder,
		aspect=1,
		draw,
		thick,
		fill,
		shape border rotate=90,
		minimum height=1.5cm,
		left color=sGreen!30,
		right color=sGreen!30,
		middle color=sGreen!30,
		minimum width=2cm,
		path picture={
			\draw[black, thick] let \p1=($(path picture bounding box.north east)-(path picture bounding
			box.south west)$) in 
			foreach \XX in {1,2,3}  {([yshift=-\XX*\y1/4]path picture bounding box.north west) 
				arc(180:360:\x1/2 and 0.25*\x1/2)};
		}}}

\makeatletter
\tikzset{
	monodatabase/.style={
		path picture={
			\draw (0, 1.5*\database@segmentheight) circle [x radius=\database@radius,y radius=\database@aspectratio*\database@radius];
			\draw (-\database@radius, 0.5*\database@segmentheight) arc [start angle=180,end angle=360,x radius=\database@radius, y radius=\database@aspectratio*\database@radius];
			\draw (-\database@radius,-0.5*\database@segmentheight) arc [start angle=180,end angle=360,x radius=\database@radius, y radius=\database@aspectratio*\database@radius];
			\draw (-\database@radius,1.5*\database@segmentheight) -- ++(0,-3*\database@segmentheight) arc [start angle=180,end angle=360,x radius=\database@radius, y radius=\database@aspectratio*\database@radius] -- ++(0,3*\database@segmentheight);
		},
		minimum width=2*\database@radius + \pgflinewidth,
		minimum height=3*\database@segmentheight + 2*\database@aspectratio*\database@radius + \pgflinewidth,
	},
	database segment height/.store in=\database@segmentheight,
	database radius/.store in=\database@radius,
	database aspect ratio/.store in=\database@aspectratio,
	database segment height=0.1cm,
	database radius=0.25cm,
	database aspect ratio=0.35,
}
\makeatother

\tikzset{
	cylinder/.style={draw,
		shape=cylinder,
		name=nodename, % Can be defined arbitrarily
		alias=cyl, % Will be used by the ellipse to reference the cylinder
		aspect=1.5,
		minimum height=8.5cm,
		minimum width=2.25cm,
		left color=blue!30,
		right color=blue!60,
		middle color=red!20, % Has to be called after left color and middle color
		outer sep=-0.5\pgflinewidth, % to make sure the ellipse does not draw over the lines
		%shape border rotate=90
}}

\tikzset{XOR/.style={draw,circle,append after command={
			[shorten >=\pgflinewidth, shorten <=\pgflinewidth,]
			(\tikzlastnode.north) edge (\tikzlastnode.south)
			(\tikzlastnode.east) edge (\tikzlastnode.west)
		}
	}
}

\newcommand{\boundellipse}[3]% center, xdim, ydim
{(#1) ellipse (#2 and #3)
}

\newcommand{\plotcurve}[3][thick, every plot/.style={smooth}]{
	% plot curve y^2 = x^3 + a x + b in range [-3,3]^2
	% parameter 1 (optional): style options for curve (color, etc)
	% parameter 2: curve parameter a
	% parameter 3: curve parameter b
	\draw[gray] (-3,-3) rectangle (3,3);
	\draw[->,>=latex,gray] (-3,0) -- (3,0);
	\draw[->,>=latex,gray] (0,-3) -- (0,3);
	\draw[name path=curve, #1] plot[id=curve#2#3, raw gnuplot] function {
		f(x,y) = y**2 - x**3 - #2*x - #3;
		set xrange [-3:3];
		set yrange [-3:3];
		set view 0,0;
		set isosample 50,50;
		set cont base;
		set cntrparam levels incre 0,0.1,0;
		unset surface;
		splot f(x,y);
	};
}

% For an explanation of the tangent coordinate system, check http://tex.stackexchange.com/a/25940 
\tikzset{
	tangent/.style={
		decoration={markings, mark=at position #1 with {
				\coordinate (tangent point-\pgfkeysvalueof{/pgf/decoration/mark info/sequence number}) at (0pt,0pt);
				\coordinate (tangent unit vector-\pgfkeysvalueof{/pgf/decoration/mark info/sequence number}) at (1,0pt);
				\coordinate (tangent orthogonal unit vector-\pgfkeysvalueof{/pgf/decoration/mark info/sequence number}) at (0pt,1);
		}},
		postaction=decorate
	},
	use tangent/.style={
		shift=(tangent point-#1),
		x=(tangent unit vector-#1),
		y=(tangent orthogonal unit vector-#1)
	},
	use tangent/.default=1
}

\usepackage{tikz-3dplot}
\newsavebox\Server
\sbox\Server{\tdplotsetmaincoords{70}{20}
	\begin{tikzpicture}[tdplot_main_coords]
		\begin{scope}[canvas is xz plane at y=3]
			\path (1,0)  coordinate(aux);
			\path (1pt,0) coordinate (BTL) (1cm-1pt,0) coordinate (BTR)
			(1cm,-2.5cm+1pt) coordinate (BBR) (1cm,-1pt) coordinate (BTR');
		\end{scope}
		\begin{scope}[canvas is xz plane at y=0]
			\draw[rounded corners={2*sqrt(2)*1pt},fill=gray!10] (0,-2.5) rectangle (1,0);
			\draw[rounded corners={2*sqrt(2)*1pt},white] (0.4pt,-2.5cm+0.4pt) 
			rectangle (1cm-0.4pt,0-0.4pt);
			\path (1pt,0) coordinate (FTL) (1cm-1pt,0) coordinate (FTR);
			\path[fill=white,rounded corners={2*sqrt(2)*1pt}]
			($(aux)+(-1,0)$) -| ++(1,-2.5) -- (1,-2.5) |- (0,0)--cycle;
			\path[left color=gray!10,right color=gray!30,rounded corners=1pt] (BTL) -- (BTR) -- (FTR) -- (FTL)
			-- cycle;
			\path[top color=gray!80,bottom color=gray!30,shading angle=20,
			rounded corners=1pt]
			(1cm,-2.5cm+1pt) -- (BBR) -- (BTR') -- (1cm,-1pt) -- cycle;
			\draw[ultra thin,fill=gray!40] foreach \X in {0.2,0.3,...,0.81} 
			{ \foreach \Y in {-0.2,-0.3,...,-2.3} 
				{(\X-0.03,\Y-0.03) rectangle (\X+0.03,\Y+0.03)}};
			\begin{scope} 
				\clip (0,-2.5) rectangle (1,-0.6pt);
				\fill[gray!10] (0.5,0) circle[radius=0.35cm];
				\shade[ball color=black!80] (0.5,0) circle[radius=0.25cm];
			\end{scope}
			\begin{scope}[rounded corners=1mm] 
				\clip (0.42,-0.7) -- (0.42,-0.9) -- (0.22,-1.1)
				-- (0.42,-1.3) -- (0.42,-2.1) -- (0.58,-2.1) 
				-- (0.58,-1.3) -- (0.78,-1.1) -- (0.58,-0.9) -- (0.58,-0.7) -- cycle;
				\fill[gray!80] (0.42,-0.7) -- (0.42,-0.9) -- (0.22,-1.1)
				-- (0.42,-1.3) -- (0.42,-2.1) -- (0.58,-2.1) 
				-- (0.58,-1.3) -- (0.78,-1.1) -- (0.58,-0.9) -- (0.58,-0.7) -- cycle;
				\fill[gray!20] (0.03+0.42,-0.7) -- (0.03+0.42,-0.9) -- (0.03+0.22,-1.1)
				-- (0.03+0.42,-1.3) -- (0.03+0.42,-2.1) -- (0.03+0.58,-2.1) 
				-- (0.03+0.58,-1.3) -- (0.03+0.78,-1.1) -- (0.03+0.58,-0.9) -- (0.03+0.58,-0.7) -- cycle;       
			\end{scope} 
			\shade[ball color=black!80] (0.5,-1.1) circle[radius=0.1cm];
		\end{scope}
	\end{tikzpicture}
}
	
\tikzset{meter/.append style={draw, inner sep=10, rectangle, font=\vphantom{A}, minimum width=30, line width=.8, path picture={\draw[black] ([shift={(.1,.3)}]path picture bounding box.south west) to[bend left=50] ([shift={(-.1,.3)}]path picture bounding box.south east);\draw[black,-latex] ([shift={(0,.1)}]path picture bounding box.south) -- ([shift={(.3,-.1)}]path picture bounding box.north);}}}

\tikzset{
	lightbulbon/.pic={
		% Bulb
		\draw[thick] (0,0) ellipse (1 and 1.5);
		% Base
		\draw[thick] (-0.5,-1.5) -- (-0.5,-2.5);
		\draw[thick] (0.5,-1.5) -- (-0.5,-1.5);
		\draw[thick] (0.5,-1.5) -- (0.5,-2.5);
		\draw[thick] (-0.5,-2.5) -- (0.5,-2.5);
		% Screws
		\draw[thick] (0.5,-1.75) -- (-0.5,-1.75);
		\draw[thick] (0.5,-2) -- (-0.5,-2);
		\draw[thick] (0.5,-2.25) -- (-0.5,-2.25);
		% Filament
		\draw[thick] (-0.3,-0.5) -- (0.3,-0.5);
		\draw[thick] (-0.3,-0.5) -- (0,-1);
		\draw[thick] (0.3,-0.5) -- (0,-1);
		% Light rays
		\foreach \angle in {0,30,...,360} {
			\draw[thick, yellow] (0,0) -- (\angle:1);
		}
	}
}

\tikzset{
	lightbulboff/.pic={
		% Bulb
		\draw[thick] (0,0) ellipse (1 and 1.5);
		% Base
		\draw[thick] (-0.5,-1.5) -- (-0.5,-2.5);
		\draw[thick] (0.5,-1.5) -- (-0.5,-1.5);
		\draw[thick] (0.5,-1.5) -- (0.5,-2.5);
		\draw[thick] (-0.5,-2.5) -- (0.5,-2.5);
		% Screws
		\draw[thick] (0.5,-1.75) -- (-0.5,-1.75);
		\draw[thick] (0.5,-2) -- (-0.5,-2);
		\draw[thick] (0.5,-2.25) -- (-0.5,-2.25);
		% Filament
		\draw[thick] (-0.3,-0.5) -- (0.3,-0.5);
		\draw[thick] (-0.3,-0.5) -- (0,-1);
		\draw[thick] (0.3,-0.5) -- (0,-1);
	}
}

\tikzset{
	capacitor/.pic={
		% Plates
		\draw[thick] (-1,0) -- (1,0);
		\draw[thick] (-1,1) -- (1,1);
		% Connecting lines
		\draw[thick] (-1,0) -- (-1,-0.5);
		\draw[thick] (1,0) -- (1,-0.5);
		\draw[thick] (-1,1) -- (-1,1.5);
		\draw[thick] (1,1) -- (1,1.5);
	}
}

\tikzset{
	photonH/.pic={
		% Plates
		\draw[<->,thick] (0,1) -- (2,1);
		\draw[<->,thick] (1,0) -- (1,2);
		\draw[<->,thick,color=kr] (0.25,1) -- (1.75,1);
	}
}

\tikzset{
	basisHV/.pic={
		% Plates
		\draw[->,thick] (0,0) -- (0:2);
		\draw[->,thick] (0,0) -- (90:2);
		\draw[->,thick] (0,0) -- (180:2);
		\draw[->,thick] (0,0) -- (270:2);
	}
}

\tikzset{
	basisAD/.pic={
		% Plates
		\draw[->,thick] (0,0) -- (45:2);
		\draw[->,thick] (0,0) -- (135:2);
		\draw[->,thick] (0,0) -- (225:2);
		\draw[->,thick] (0,0) -- (315:2);
	}
}

\tikzset{
	photonV/.pic={
		% Plates
		\draw[<->,thick] (0,1) -- (2,1);
		\draw[<->,thick] (1,0) -- (1,2);
		\draw[<->,thick,color=sBlue] (1,0.25) -- (1,1.75);
	}
}

\tikzset{
	photonA/.pic={
		% Plates
		\draw[<->,thick] (0,1) -- (2,1);
		\draw[<->,thick] (1,0) -- (1,2);
		\draw[<->,thick,color=sGreen] (0.3,1.6) -- (1.6,0.3);
	}
}

\tikzset{
	photonD/.pic={
		% Plates
		\draw[<->,thick] (0,1) -- (2,1);
		\draw[<->,thick] (1,0) -- (1,2);
		\draw[<->,thick,color=sPurple] (0.3,0.3) -- (1.6,1.6);
	},
	polarglass/.pic={
		% Define cube vertices
		\coordinate (A) at (2,0);
		\coordinate (B) at (3,0);
		\coordinate (C) at (3,1);
		\coordinate (D) at (2,1);
		\coordinate (E) at (2.25,0.25);
		\coordinate (F) at (3.25,0.25);
		\coordinate (G) at (3.25,1.25);
		\coordinate (H) at (2.25,1.25);
		
		% Draw front face
		\draw[thick] (A) -- (B) -- (C) -- (D) -- cycle;
		
		% Draw top face
		\draw[thick] (D) -- (H) -- (G) -- (C);
		
		% Draw side face
		\draw[thick] (C) -- (G) -- (F) -- (B);
	}
}

\tikzset{meter/.append style={draw, inner sep=10, rectangle, font=\vphantom{A}, minimum width=30, line width=.8,
		path picture={\draw[black] ([shift={(.1,.3)}]path picture bounding box.south west) to[bend left=50] ([shift={(-.1,.3)}]path picture bounding box.south east);\draw[black,-latex] ([shift={(0,.1)}]path picture bounding box.south) -- ([shift={(.3,-.1)}]path picture bounding box.north);}}}
	
\tikzset{
	cylinder/.style={draw,
		shape=cylinder,
		name=nodename, % Can be defined arbitrarily
		alias=cyl, % Will be used by the ellipse to reference the cylinder
		aspect=1.5,
		minimum height=8.5cm,
		minimum width=2.25cm,
		left color=blue!30,
		right color=blue!60,
		middle color=red!20, % Has to be called after left color and middle color
		outer sep=-0.5\pgflinewidth, % to make sure the ellipse does not draw over the lines
		%shape border rotate=90
}}

\tikzset{database/.style={
		cylinder,
		aspect=1,
		draw,
		thick,
		fill,
		shape border rotate=90,
		minimum height=1cm,
		left color=black!30,
		right color=black!30,
		middle color=black!30,
		minimum width=1cm,
		path picture={
			\draw[black, thick] let \p1=($(path picture bounding box.north east)-(path picture bounding
			box.south west)$) in
			foreach \XX in {1,2,3}  {([yshift=-\XX*\y1/4]path picture bounding box.north west)
				arc(180:360:\x1/2 and 0.25*\x1/2)};
}}}

\newcommand{\data}[5]{\draw [#1] (#3,#2) -- ++(#4-#3,0) node[midway,above] {\small #5} ;}
% Include Tikz images
\newcommand{\inputTikz}[2][1]{%
	\centering
	\scalebox{#1}{
		\input{#2.tikz}%
}}